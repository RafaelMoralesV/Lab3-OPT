% Template:     Informe LaTeX
% Documento:    Configuraciones del template
% Versión:      8.1.6 (12/06/2022)
% Codificación: UTF-8
%
% Autor: Pablo Pizarro R.
%        pablo@ppizarror.com
%
% Manual template: [https://latex.ppizarror.com/informe]
% Licencia MIT:    [https://opensource.org/licenses/MIT]

% CONFIGURACIONES GENERALES
\def\documentfontsize {11}         % Tamaño de la fuente del documento [pt]
\def\documentinterline {1.025}     % Interlineado del documento [factor]
\def\documentparindent {15}        % Tamaño del indentado de párrafos [pt]
\def\documentparskip {0}           % Tamaño adicional entre párrafos (+/-) [pt]
\def\fontdocument {lmodern}        % Tipografía base, ver soportadas en manual
\def\fonttypewriter {tmodern}      % Tipografía de \texttt, ver manual
\def\fonturl {same}                % Tipo de fuente url {tt,sf,rm,same}
\def\graphicxdraft {false}         % En true no carga las imágenes (modo draft)
\def\indentfirstpar {false}        % Obliga la indentación del primer párrafo
\def\pointdecimal {true}           % N° decimales con punto en vez de coma
\def\predocpageromannumber {true}  % Pág. con número romano previo a inicio doc.
\def\predocpageromanupper {false}  % Páginas en número romano en mayúsculas
\def\predocresetpagenumber {true}  % Resetea número de la página tras el índice
\def\showlayoutlines {false}       % Muestra el layout de la página
\def\showlinenumbers {false}       % Muestra los números de línea del documento
\def\twopagesclearformat {blank}   % Formato nueva página twoside {blank,empty}

% ESTILO PORTADA Y HEADER-FOOTER
\def\disablehfrightmark {false}    % Desactiva el rightmark del header-footer
\def\hfstyle {style1}              % Estilo header-footer (16 estilos)
\def\hfwidthcourse {0.35}          % Tamaño máximo del curso en header-footer
\def\hfwidthtitle {0.6}            % Tamaño máximo del título en header-footer
\def\hfwidthwrap {false}           % Activa el tamaño máximo en header-footer
\def\portraitstyle {style1}        % Estilo portada (21 estilos)

% CONFIGURACIÓN DE LAS LEYENDAS - CAPTION
\def\captionalignment {justified}  % Posición {centered,justified,left,right}
\def\captionfontsize {small}       % Tamaño de fuente de los caption
\def\captionlabelformat {simple}   % Formato leyenda {empty,simple,parens}
\def\captionlabelsep {colon}       % Sep. {none,colon,period,space,quad,newline}
\def\captionlessmarginimage {0.1}  % Margen sup/inf de figura sin leyenda [cm]
\def\captionlrmargin {2}           % Márgenes izq/der de la leyenda [cm]
\def\captionlrmarginmc {1}         % Margen izq/der leyenda dentro de cols. [cm]
\def\captionmarginimage {0}        % Margen vertical entre caption e imagen [cm]
\def\captionmarginimages {0}       % Margen vertical entre caption e images [cm]
\def\captionmarginimagesmc {0}     % Margen vert. entre caption e imagesmc [cm]
\def\captionmarginmultimg {0}      % Margen izq/der leyendas múltiple img [cm]
\def\captionnumcode {arabic}       % N° código {arabic,alph,Alph,roman,Roman}
\def\captionnumequation {arabic}   % N° ecuac. {arabic,alph,Alph,roman,Roman}
\def\captionnumfigure {arabic}     % N° figuras {arabic,alph,Alph,roman,Roman}
\def\captionnumsubfigure {alph}    % N° subfigs. {arabic,alph,Alph,roman,Roman}
\def\captionnumsubtable {alph}     % N° subtabla {arabic,alph,Alph,roman,Roman}
\def\captionnumtable {arabic}      % N° tabla {arabic,alph,Alph,roman,Roman}
\def\captionsubchar {.}            % Caracter entre N° objeto - subfigura/tabla
\def\captiontbmarginfigure {9.35}  % Margen sup/inf de leyenda en figuras [pt]
\def\captiontbmargintable {7}      % Margen sup/inf de la leyenda en tablas [pt]
\def\captiontextbold {false}       % Etiqueta (código,figura,tabla) en negrita
\def\captiontextsubnumbold {false} % N° subfigura/subtabla en negrita
\def\codecaptiontop {true}         % Leyenda arriba del código fuente
\def\equationcaptioncenter {true}  % Ecuaciones están centradas o justificadas
\def\figurecaptiontop {false}      % Leyenda arriba de las imágenes
\def\marginaligncaptbottom {0.1}   % Margen inferior caption en align [cm]
\def\marginaligncapttop {-0.75}    % Margen superior caption en align [cm]
\def\marginalignedcaptbottom {0.1} % Margen inferior caption en aligned [cm]
\def\marginalignedcapttop {-0.75}  % Margen superior caption en aligned [cm]
\def\margineqncaptionbottom {0}    % Margen inferior caption ecuación [cm]
\def\margineqncaptiontop {-0.7}    % Margen superior caption ecuación [cm]
\def\margingathercaptbottom {0.1}  % Margen inferior caption en gather [cm]
\def\margingathercapttop {-0.9}    % Margen superior caption en gather [cm]
\def\margingatheredcaptbottom{0.1} % Margen inferior caption en gathered [cm]
\def\margingatheredcapttop {-0.7}  % Margen superior caption en gathered [cm]
\def\sectioncaptiondelimiter {.}   % Caracter delimitador n° objeto y sección
\def\showsectioncaptioncode {none} % N° sec código {none,chap,(s/ss/sss/ssss)ec}
\def\showsectioncaptioneqn {none}  % N° sec ecuac. {none,chap,(s/ss/sss/ssss)ec}
\def\showsectioncaptionfig {none}  % N° sec figs. {none,chap,(s/ss/sss/ssss)ec}
\def\showsectioncaptionmat {none}  % N° matemático {none,chap,(s/ss/sss/ssss)ec}
\def\showsectioncaptiontab {none}  % N° sec tablas {none,chap,(s/ss/sss/ssss)ec}
\def\subcaptionfsize{footnotesize} % Tamaño de la fuente de los subcaption
\def\subcaptionlabelformat{parens} % Formato leyenda sub. {empty,simple,parens}
\def\subcaptionlabelsep {space}    % Sep. {none,colon,period,space,quad,newline}
\def\tablecaptiontop {true}        % Leyenda arriba de las tablas

% CONFIGURACIÓN DEL ÍNDICE
\def\addindexsubtobookmarks{false} % Agrega índice tabla,codigo,etc a marcadores
\def\addindextobookmarks {true}    % Añade el índice a los marcadores del pdf
\def\charafterobjectindex {.}      % Caracter después de n° figura,tabla,código
\def\charnumpageindex {.}          % Caracter número de página en índice
\def\indexdepth {4}                % Profundidad máxima del índice
\def\indexnewpagec {false}         % Nueva página en índice códigos fuente
\def\indexnewpagee {false}         % Nueva página en índice ecuaciones
\def\indexnewpagef {false}         % Nueva página en índice figuras
\def\indexnewpaget {false}         % Nueva página en índice tablas
\def\indexstyle {ftc}              % Tipo {f:figura,t:tabla,c:código,e:ecuación}
\def\indextitlemargin {11.4}       % Margen título índice \insertindextitle [pt]
\def\objectindexindent {false}     % Indenta la lista de objetos
\def\showappendixsecindex {false}  % Título de la sección de anexos en el índice
\def\showindexofcontents {true}    % Activa la lista de contenidos en el índice

% ANEXO, CITAS, REFERENCIAS
\def\apacitebothers {et al.}       % Etiqueta usada en (y otros) con \shortcite
\def\apaciterefcitecharclose {]}   % Caracter final cita apacite
\def\apaciterefcitecharopen {[}    % Caracter inicial cita apacite
\def\apaciterefnumber {false}      % Lista de referencias con números
\def\apaciterefsep {5}             % Separación entre refs. {apacite} [pt]
\def\apaciteshowurl {false}        % Muestra las url en las referencias
\def\apacitestyle {apacite}        % Formato refs. apacite {apa,ieeetr,etc..}
\def\appendixindepobjnum {true}    % Anexo usa n° objetos independientes
\def\backrefpagecite {false}       % Las citas en bibliografía poseen nº de pag.
\def\bibtexenvrefsecnum {false}    % Entorno references es una sección numerada
\def\bibtexindexbibliography{true} % Función \bibliography se inserta en índice
\def\bibtexrefsep {5}              % Separación entre refs. {bibtex} [pt]
\def\bibtexstyle {ieeetr}          % Formato refs. bibtex {apa,ieeetr,etc...}
\def\bibtextextalign {justify}     % Alineac. bibtex {justify,left,right,center}
\def\fontsizerefbibl {\normalsize} % Tamaño fuente al usar \bibliography{file}
\def\natbibrefcitecharclose {]}    % Caracter final cita natbib
\def\natbibrefcitecharopen {[}     % Caracter inicial cita natbib
\def\natbibrefcitecompress {true}  % Comprime refererencias al citar
\def\natbibrefcitesepcomma {true}  % Separador en coma (,) o punto y coma (;)
\def\natbibrefcitetype {numbers}   % Tipo citación {authoryear,numbers,super}
\def\natbibrefsep {5}              % Separación entre referencia {natbib} [pt]
\def\natbibrefstyle {natnumurl}    % Formato refs. natbib {apa,ieeetr,etc...}
\def\stylecitereferences {natbib}  % Estilo refs. {apacite,bibtex,natbib,custom}
\def\twocolumnreferences {false}   % Referencias en dos columnas

% CONFIGURACIONES DE OBJETOS
\def\animatedimageautoplay {true}  % Autoplay en imágenes animadas
\def\animatedimagecontrols {false} % Muestra los controles en imágenes animadas
\def\animatedimageloop {true}      % Hace loops en imágenes animadas
\def\columnsepwidth {2.1}          % Separación entre columnas [em]
\def\defaultimagefolder {img/}     % Carpeta raíz de las imágenes
\def\equationleftalign {false}     % Ecuaciones alineadas a la izquierda
\def\equationrestart {none}        % Reinicio n° {none,chap,(s/ss/sss/ssss)ec}
\def\footnotelmargin {10}          % Margen entre footnote y el número [pt]
\def\footnoterestart {none}        % N° foot. {none,chap,page,(s/ss/sss/ssss)ec}
\def\footnoterulefigure {false}    % Footnote en figuras tienen línea superior
\def\footnoterulepage {false}      % Footnote en páginas tienen línea superior
\def\footnoteruletable {false}     % Footnote en tablas tienen línea superior
\def\footnotetwocolumn {false}     % Footnote en dos columnas
\def\fpremovetopbottomcenter{true} % Elimina espacio vert. al centrar con b!,t!
\def\hfpdashcharstyle {--}         % Estilo guión en portadas,header-footer,etc
\def\imagedefaultplacement {H}     % Posición por defecto de las imágenes
\def\marginalignbottom {-0.4}      % Margen inferior entorno align [cm]
\def\marginalignedbottom {-0.2}    % Margen inferior entorno aligned [cm]
\def\marginalignedtop {-0.4}       % Margen superior entorno aligned [cm]
\def\marginaligntop {-0.4}         % Margen superior entorno align [cm]
\def\margineqnindexbottom {-0.9}   % Margen inferior ecuaciones índice [cm]
\def\margineqnindextop {0}         % Margen superior ecuaciones índice [cm]
\def\marginequationbottom {-0.2}   % Margen inferior ecuaciones [cm]
\def\marginequationtop {0}         % Margen superior ecuaciones [cm]
\def\marginfloatimages {-13}       % Margen sup figs. insertimageleft/right [pt]
\def\margingatherbottom {-0.2}     % Margen inferior entorno gather [cm]
\def\margingatheredbottom {-0.1}   % Margen inferior entorno gathered [cm]
\def\margingatheredtop {-0.4}      % Margen superior entorno gathered [cm]
\def\margingathertop {-0.4}        % Margen superior entorno gather [cm]
\def\marginimagebottom {-0.15}     % Margen inferior figura [cm]
\def\marginimagemultbottom {-0.05} % Margen inferior imágenes múltiples [cm]
\def\marginimagemultright {0.5}    % Margen derecho imágenes múltiples [cm]
\def\marginimagemulttop {-0.3}     % Margen superior imágenes múltiples [cm]
\def\marginimagetop {0}            % Margen superior figuras [cm]
\def\marginlinenumbers {7.5}       % Margen izquierdo (\showlinenumbers) [pt]
\def\numberedequation {true}       % Ecuaciones con \insert... numeradas
\def\senumerti {\arabic{enumi}.}   % Estilo enumerate nivel 1
\def\senumertii {\alph{enumii})}   % Estilo enumerate nivel 2
\def\senumertiii{\roman{enumiii}.} % Estilo enumerate nivel 3
\def\senumertiv {\Alph{enumiv})}   % Estilo enumerate nivel 4
\def\sitemizei {\iitembcirc}       % Estilo itemize nivel 1
\def\sitemizeii {\iitemdash}       % Estilo itemize nivel 2
\def\sitemizeiii {\iitemcirc}      % Estilo itemize nivel 3
\def\sitemizeiv {\iitembsquare}    % Estilo itemize nivel 4
\def\sitemsmargini {25}            % Margen ítems nivel 1 [pt]
\def\sitemsmarginii {22}           % Margen ítems nivel 2 [pt]
\def\sitemsmarginiii {18.7}        % Margen ítems nivel 3 [pt]
\def\sitemsmarginiv {17}           % Margen ítems nivel 4 [pt]
\def\sourcecodebgmarginbottom {0}  % Margen inferior del bloque de color [pt]
\def\sourcecodebgmarginleft {0}    % Margen izquierdo del bloque de color [pt]
\def\sourcecodebgmarginright {0}   % Margen derecho del bloque de color [pt]
\def\sourcecodebgmargintop {0}     % Margen superior del bloque de color [pt]
\def\sourcecodefontf {\ttfamily}   % Tipo de letra código fuente
\def\sourcecodefonts {\small}      % Tamaño letra código fuente
\def\sourcecodeilfontf {\ttfamily} % Tipo de letra código fuente inline
\def\sourcecodeilfonts {\small}    % Tamaño letra código fuente inline
\def\sourcecodenumbersep {6}       % Separación entre número línea y código [pt]
\def\sourcecodenumbersize {\tiny}  % Tamaño fuente número línea
\def\sourcecodeskipabove {0.75}    % Espacio sobre recuadro código [em]
\def\sourcecodeskipbelow {0.95}    % Espacio bajo recuadro código [em]
\def\sourcecodetabsize {3}         % Tamaño tabulación código fuente
\def\tabledefaultplacement {H}     % Posición por defecto de las tablas
\def\tablenotesameline {true}      % Notas en tablas en una sola línea
\def\tablenotesfontsize {\small}   % Tamaño de fuente de las notas en tablas
\def\tablepaddingh {0.75}          % Espaciado horizontal de celda de las tablas
\def\tablepaddingv {1.15}          % Espaciado vertical de celda de las tablas
\def\tikzdefaultplacement {H}      % Posición por defecto de las figuras tikz

% CONFIGURACIÓN DE LOS TÍTULOS
\def\anumsecaddtocounter {false}   % Insertar títulos anum. aumenta n° de sec
\def\charaftersectionnum {.}       % Caracter después n° (s/ss/sss/ssss)ection
\def\charappendixsection {.}       % Caracter entre n° sección anexo y título
\def\charbetwchaptersection {.}    % Caracter entre nº capítulo y sección
\def\charbetwsectionsubsection {.} % Caracter entre nº sección y subsección
\def\charbetwssectionsssect {.}    % Caracter entre nº subsubsección y sssección
\def\charbetwsubsectionssect {.}   % Caracter entre nº subsección y ssección
\def\formatnumapchapter {\arabic}  % Formato nº capítulo en appendixd
\def\formatnumapsection {\Alph}    % Formato nº sección en appendixd
\def\formatnumapssection {\arabic} % Formato nº subsección en appendixd
\def\formatnumapsssection{\arabic} % Formato nº sub-subsección en appendixd
\def\formatnumapssssection{\arabic}% Formato nº sub-sub-subsección appendixd
\def\formatnumchapter {\arabic}    % Formato número capítulo
\def\formatnumsection {\arabic}    % Formato número sección
\def\formatnumssection {\arabic}   % Formato número subsección
\def\formatnumsssection {\arabic}  % Formato número sub-subsección
\def\formatnumssssection {\arabic} % Formato número sub-sub-subsección
\def\indexsectionfontsize {\LARGE} % Tamaño section en el índice
\def\indexsectionstyle {\bfseries} % Estilo section en el índice
\def\paragfontsize {\normalsize}   % Tamaño fuente paragraph
\def\paragfontstyle {\bfseries}    % Estilo fuente paragraph
\def\paragspacingbottom {4}        % Espaciado inferior en paragraph [pt]
\def\paragspacingleft {0}          % Espaciado izq. en paragraph [pt]
\def\paragspacingtop {8}           % Espaciado superior en paragraph [pt]
\def\paragsubfontsize{\normalsize} % Tamaño fuente subparagraph
\def\paragsubfontstyle {\bfseries} % Estilo fuente subparagraph
\def\paragsubspacingbottom {4}     % Espaciado inferior en subparagraph [pt]
\def\paragsubspacingleft {0}       % Espaciado izq. en subparagraph [pt]
\def\paragsubspacingtop {8}        % Espaciado superior en subparagraph [pt]
\def\sectionfontsize {\LARGE}      % Tamaño fuente section
\def\sectionfontstyle {\bfseries}  % Estilo fuente section
\def\sectionspacingbottom {10}     % Espaciado inferior en section [pt]
\def\sectionspacingleft {0}        % Espaciado izq. en section [pt]
\def\sectionspacingtop {20}        % Espaciado superior en section [pt]
\def\spacingaftersection {\quad}   % Espaciador después nº sección
\def\ssectionfontsize {\Large}     % Tamaño fuente subtítulos
\def\ssectionfontstyle {\bfseries} % Estilo fuente subsection
\def\ssectionspacingbottom {10}    % Espaciado inferior en subsection [pt]
\def\ssectionspacingleft {0}       % Espaciado izq. en subsection [pt]
\def\ssectionspacingtop {12}       % Espaciado superior en subsection [pt]
\def\sssectionfontsize {\large}    % Tamaño fuente subsubsection
\def\sssectionfontstyle{\bfseries} % Estilo fuente subsubsection
\def\sssectionspacingbottom {8}    % Espaciado inferior en subsubsection [pt]
\def\sssectionspacingleft {0}      % Espaciado izq. en subsubsection [pt]
\def\sssectionspacingtop {10}      % Espaciado superior en subsubsection [pt]
\def\ssssectionfontstyle{\bfseries}% Estilo fuente subsubsubsection
\def\ssssectionfontsz{\normalsize} % Tamaño fuente subsubsubsection
\def\ssssectionspacingbottom {6}   % Espaciado inferior en subsubsubsection [pt]
\def\ssssectionspacingleft {0}     % Espaciado izq. en subsubsubsection [pt]
\def\ssssectionspacingtop {8}      % Espaciado superior en subsubsubsection [pt]

% CONFIGURACIÓN DE LOS COLORES DEL DOCUMENTO
\def\captioncolor {black}          % Color nombre objeto (código,figura,tabla)
\def\captiontextcolor {black}      % Color de la leyenda
\def\enumerateitemcolor {black}    % Color de los enumerate por defecto
\def\highlightcolor {yellow}       % Color del subrayado con \hl
\def\indextitlecolor {black}       % Color de los títulos del índice
\def\itemizeitemcolor {black}      % Color de los ítems por defecto
\def\linenumbercolor {gray}        % Color del n° de línea (\showlinenumbers)
\def\linkcolor {black}             % Color de los links del documento
\def\maintextcolor {black}         % Color principal del texto
\def\numcitecolor {black}          % Color del n° de las referencias o citas
\def\pagescolor {white}            % Color de la página
\def\paragcolor {black}            % Color de los paragraph
\def\paragsubcolor {black}         % Color de los subparagraph
\def\portraittitlecolor {black}    % Color de los títulos de la portada
\def\sectioncolor {black}          % Color de los section
\def\showborderonlinks {false}     % Color de un link por un recuadro de color
\def\sourcecodebgcolor {lgray}     % Color de fondo del código fuente
\def\ssectioncolor {black}         % Color de los subsection
\def\sssectioncolor {black}        % Color de los subsubsection
\def\ssssectioncolor {black}       % Color de los subsubsubsection
\def\tablelinecolor {black}        % Color de las líneas de las tablas
\def\tablerowfirstcolor {none}     % Primer color de celda de las tablas
\def\tablerowsecondcolor {gray!20} % Segundo color de celda de las tablas
\def\urlcolor {magenta}            % Color de los enlaces web (\href,\url)

% MÁRGENES DE PÁGINA
\def\firstpagemargintop {3.8}      % Margen superior página portada [cm]
\def\pagemarginbottom {2.7}        % Margen inferior página [cm]
\def\pagemarginleft {2.54}         % Margen izquierdo página [cm]
\def\pagemarginright {2.54}        % Margen derecho página [cm]
\def\pagemargintop {3}             % Margen superior página [cm]

% OPCIONES DEL PDF COMPILADO
\def\cfgbookmarksopenlevel {1}     % Nivel marcadores en pdf (1:secciones)
\def\cfgpdfbookmarkopen {true}     % Expande marcadores del nivel configurado
\def\cfgpdfcenterwindow {true}     % Centra ventana del lector al abrir el pdf
\def\cfgpdfcopyright {}            % Establece el copyright del documento
\def\cfgpdfdisplaydoctitle {true}  % Muestra título del informe en visor
\def\cfgpdffitwindow {false}       % Ajusta la ventana del lector tamaño pdf
\def\cfgpdfkeywords {}             % Palabras clave del pdf
\def\cfgpdflayout {OneColumn}      % Modo de página {OneColumn,SinglePage}
\def\cfgpdfmenubar {true}          % Muestra el menú del lector
\def\cfgpdfpageview {FitH}         % {Fit,FitH,FitV,FitR,FitB,FitBH,FitBV}
\def\cfgpdfsecnumbookmarks {true}  % Número de la sec. en marcadores del pdf
\def\cfgpdftoolbar {true}          % Muestra barra de herramientas lector pdf
\def\cfgshowbookmarkmenu {false}   % Muestra menú marcadores al abrir el pdf
\def\pdfcompilecompression {9}     % Factor de compresión del pdf (0-9)
\def\pdfcompileobjcompression {2}  % Nivel compresión objetos del pdf (0-3)
\def\pdfcompileversion {7}         % Versión mínima del pdf compilado
\def\usepdfmetadata {true}         % Añade metadatos al pdf compilado

% NOMBRE DE OBJETOS
\def\nameabstract {Resumen}           % Nombre del resumen-abstract
\def\nameappendixsection {Anexos}     % Nombre de los anexos
\def\namechapter {Capítulo}           % Nombre de los capítulos
\def\nameltappendixsection {Anexo}    % Etiqueta sección en anexo/apéndices
\def\nameltcont{Índice de Contenidos} % Nombre del índice de contenidos
\def\namelteqn {Índice de Ecuaciones} % Nombre de la lista de ecuaciones
\def\nameltfigure {Índice de Figuras} % Nombre de la lista de figuras
\def\nameltsrc {Índice de Códigos}    % Nombre de la lista de código
\def\namelttable {Índice de Tablas}   % Nombre de la lista de tablas
\def\nameltwfigure {Figura}           % Etiqueta leyenda de las figuras
\def\nameltwsrc {Código}              % Etiqueta leyenda del código fuente
\def\nameltwtable {Tabla}             % Etiqueta leyenda de las tablas
\def\namemathcol {Corolario}          % Nombre de los colorarios
\def\namemathdefn {Definición}        % Nombre de las definiciones
\def\namemathej {Ejemplo}             % Nombre de los ejemplos
\def\namemathlem {Lema}               % Nombre de los lemas
\def\namemathobs {Observación}        % Nombre de las observaciones
\def\namemathprp {Proposición}        % Nombre de las proposiciones
\def\namemaththeorem {Teorema}        % Nombre de los teoremas
\def\namepageof { de }                % Etiqueta página # de #
\def\nameportraitpage {Cover}         % Etiqueta página de la portada
\def\namereferences {Referencias}     % Nombre de la sección de referencias
