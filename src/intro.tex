\section{Introducción}
\lipsum[3]

\subsection{Problema presentado}
Una empresa constructora ha comprado un terreno para construir edificios. Los departamentos tendrán un precio entre los $300,000$ y los $1,200,000$ US\$.

La empresa desarrolló los planes para tres tamaños de proyecto; 6 pisos con 30 departamentos, 12 pisos con 60 departamentos y 18 pisos con 90 departamentos.

La empresa debe tomar una de las 3 decisiones:

\begin{enumerate}
    \item Edificio pequeño de 6 pisos y 30 departamentos
    \item Edificio pequeño de 12 pisos y 60 departamentos
    \item Edificio pequeño de 18 pisos y 90 departamentos
\end{enumerate}

La empresa está considerando dos posibilidades de aceptación de mercado del proyecto.

\begin{enumerate}
    \item Alta aceptación del mercado.
    \item Baja aceptación del mercado.
\end{enumerate}

La empresa ha recabado información sobre la utilidad asociada a cada una de las alternativas de decisión y estados de la naturaleza.

La tabla de pagos muestra las utilidades asociadas en millones de dólares.

\begin{table}[H]
    \begin{tabular}{r|cc|}
        \cline{2-3}
        \multicolumn{1}{l|}{}                                   & \multicolumn{2}{l|}{\textbf{Estado de la naturaleza}}                                     \\ \hline
        \multicolumn{1}{|l|}{\textbf{Alternativas de Decisión}} & \multicolumn{1}{l|}{Alta Demanda}                     & \multicolumn{1}{l|}{Baja Demanda} \\ \hline
        \multicolumn{1}{|r|}{Edif. Pequeño}                     & \multicolumn{1}{c|}{8}                                & 7                                 \\
        \multicolumn{1}{|r|}{Edif. Mediano}                     & \multicolumn{1}{c|}{14}                               & 5                                 \\
        \multicolumn{1}{|r|}{Edif. Grande}                      & \multicolumn{1}{c|}{20}                               & -9                                \\ \hline
    \end{tabular}
\end{table}

¿Qué tamaño debería seleccionarse?