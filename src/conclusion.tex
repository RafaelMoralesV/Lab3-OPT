\section{Conclusión}
Es posible realizar la comparación entre software y resolución analítica para verificar que los resultados son correctos. Sin embargo, algo que llama la atención es como no hay ningún acuerdo entre criterios por cual decisión debería tomarse.

Estas diferencias son comunes, debido a que cada estrategia de cálculo prioriza alternativas radicalmente distintas. Mientras que algunas intentan lograr el valor más alto posible, otras intentas mantener la menor pérdida.

Las diferencias fundamentales en lo que cada criterio avalúa hace que este caos de decisiones correctas abunden en muchos problemas, por lo que la última decisión siempre debería ser tomada de acuerdo a lo que la persona que debe tomarla avalúe como más importante, lo que finalmente se traduce a tomar la decisión de acuerdo al criterio que más sentido le haga a este individuo.
